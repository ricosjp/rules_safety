\documentclass[10pt,a4paper,uplatex]{jsarticle}
\usepackage{bm}
\usepackage{graphicx}
\usepackage[truedimen,left=25truemm,right=25truemm,top=25truemm,bottom=25truemm]{geometry}
\usepackage{array}
\usepackage{titlesec}
\usepackage{jpdoc}
\usepackage{tree}
\usepackage[nolists,nomarkers]{endfloat}

\renewcommand{\tablename}{表}
\renewcommand{\figurename}{図}

\titleformat*{\section}{\large\bfseries}
\def\title{自家用電気工作物保安規程}

\begin{document}
	\newpage
{\centering \Large\bf \title  \vskip 0em}
\vskip 2em


\subsection{総則}
\article{目的}
この規程は、電気事業法(昭和39年法律第170号)第42条第1項及び同法施行規則第50条第1項の規定に基づき、電気工作物の工事等(以下「電気工作物の工事等」という。)における保安の確保に関し必要な事項を定めるものとする。
\article{効力}
当社の電気工作物の工事等の業務に従事する者(以下「業務従事者」という。)は、電気関係法令及びこの規程を遵守するものとする。

\article{細則の制定}
この規程を実施するため必要と認められる場合には、別に細則を定めるものとする。
\article{規定等の改正}
この規程の改正または前条に定める細則の制定あるいは改正にあたっては、電気主任技術者の参画のもとに立案し、これを決定するものとする。

\subsection{保安業務の運営管理体制}

\article{保安業務の組織}
電気工作物の工事等に関する責任の所在を明確にし、並びに指揮命令系統及び連絡系統を明確にするため、電気工作物の工事等に関する保安業務を遂行する組織構成は次に定めるところによるものとする。
\begin{enumerate}
  \itm 電気工作物の工事等に関する保安業務を総括するため、総括管理者を置き、代表取締役をもって充てる。
  \itm 法令及びこの規程に基づく保安監督の職務を的確に遂行するために電気主任技術者を置く。
\end{enumerate}

\article{設置者の義務}
電気工作物に関する保安上重要な事項を決定又は行おうとするときは、電気主任技術者の意見を求めるものとする。
\term 電気主任技術者の電気工作物に係る保安に関する意見を尊重するものとする。
\term 法令に基づいて所管官庁に提出する書類の内容が電気工作物の保安に関係のある場合には、電気主任技術者の参画のもとに立案し、決定するものとする。
\term 所管官庁が法令に基づいて行う検査には、電気主任技術者を立ち合わせるものとする。

\article{電気主任技術者の義務}
電気主任技術者は、総括管理者を補佐し、電気工作物の工事等に関する保安の監督の職務を総括しなければならない。
\term 電気主任技術者は、法令及びこの規程を遵守し、電気工作物の工事等に関する保安の監督の職務を誠実に行わなければならない。
%\term 電気主任技術者の執務は次の各号に定めるところにより行うものとする。
%\begin{enumerate}
%  \itm 執務する回数は、電気工作物の設置、改造等の工事期間中は毎週1回以上、その他の場合にあっては毎月1回以上とする。
%  \itm 執務する時間は1回につき4時間以上とする。
%  \itm 電気主任技術者の常時勤務する場所及び連絡方法については、受電室その他見やすい箇所に掲示しておくとともに、電気主任技術者との連絡責任者を選任しておくものとする。
%\end{enumerate}

\article{従事者の義務}
電気工作物の工事、維持又は運用に従事する者は、電気主任技術者がその保安のためにする指示に従わなければならない。
\article{電気主任技術者不在時の措置}
電気主任技術者が病気その他やむを得ない事情により不在となる場合には、その業務の代行を行う者(以下「代務者」という。)をあらかじめ指名しておくものとする。
\term 代務者は、電気主任技術者の不在時には、電気主任技術者に指示された職務を誠実に行わなければならない。

\article{電気主任技術者の解任}
電気主任技術者が次の各号に該当する場合は、解任することができるものとする。
\begin{enumerate}
  \itm 電気主任技術者が病気等により欠勤が長期にわたり、保安の確保上不適当と認められたとき。
  \itm 電気主任技術者が法令又は、この規程の定めるところに違反し、又は怠って保安の確保上不適当と認められたとき。
  \itm 電気主任技術者の常時勤務する場所及び連絡方法については、受電室その他見やすい箇所に掲示しておくとともに、電気主任技術者との連絡責任者を選任しておくものとする。
\end{enumerate}
 
\subsection{保安教育}
\article{保安教育}
総括管理者は、業務従事者に対し、電気工作物の工事等の保安に関し必要な知識及び技能の教育を計画的に行うものとする。
\article{保安に関する訓練}
総括管理者は、必要に応じて業務従事者に対し、電気事故その他非常災害の発生時の措置に関する実地指導訓練を行うものとする。

\subsection{工事の計画及び実施}
\article{工事計画}
電気工作物の設置、改造等の工事計画を立案するにあたっては、電気主任技術者の意見を求めるものとする。
\article{工事の実施}
電気工作物に関する工事の実施に当たっては、その工事の内容に応じ作業責任者を選任し、主任技術者の監督の下に、これを施工するものとする。

\subsection{保守}
\article{巡視、点検、測定}
電気工作物の保安のための巡視、点検及び測定は、表\ref{日常巡視点検手入項目}、表\ref{定期巡視点検手入項目}、表\ref{精密点検手入項目}及び表\ref{測定項目}に定める基準により行わなければならない。
\term 電気主任技術者は、表\ref{日常巡視点検手入項目}、表\ref{定期巡視点検手入項目}、表\ref{精密点検手入項目}及び表\ref{測定項目}に定める基準により電気工作物の保守業務の指導監督を行うにあたっては、当事業場の営業活動等と調整を図り年度実施計画を作成し、総括管理者の承認を経てこれを実施しなければならない。
\term 巡視、点検又は測定の結果、法令に定める技術基準に適合しない事項が判明したときには当該電気工作物を修理し、改造し、移設し又はその使用を一時停止し、若しくは制限する等の措置を講じ常に技術基準に適合するよう維持するものとする。

\begin{table}[!b]
\scriptsize
  \caption{日常巡視点検手入項目}
  \label{日常巡視点検手入項目}
  \begin{tabular}{|l|c|p{1cm}|p{10cm}|} \hline
    対象         & & 周期 & 点検箇所と目的  \\ \hline
受変電設備
    &電線及び支持物  & 1ヶ月 & 電線の高さ及び他の工作物樹木との離隔距離  \\
    &              & 1ヶ月 & 標識保護さくの状況  \\ \cline{2-4}
    &ケーブル       & 1ヶ月 & ヘッド、接続箱、分岐箱など接続部の加熱、損傷、腐食及びコンパウンド油漏れ  \\
    &              & 1ヶ月 & 布設部の無断掘削  \\ 
    &              & 1ヶ月 & 標識、他物との離隔距離  \\ \cline{2-4}
    &断路器         & 1ヶ月 & 受けと刃の接触、過熱、変色、ゆるみ  \\
    &              & 1ヶ月 & 汚損、異物付着  \\
    &              & 1ヶ月 & その他必要事項  \\\cline{2-4}
    &遮断器         & 1ヶ月 & 外観点検、汚損、油洩れ、きれつ、過熱、発錆、損傷  \\
    &開閉器類       & 1ヶ月 & 指示、点灯  \\ 
    &              & 1ヶ月 & その他必要事項  \\ \cline{2-4}
    &母線          & 1ヶ月 & 必要により特定部位のものについて行う(点検箇所、ねらいは定期巡視点検より抜粋)  \\ \cline{2-4}
    &受電用変圧器   & 1ヶ月 & 本体の外部点検、漏油、損傷、汚損、変形、ゆるみ、発錆、腐食、振動、音響、油量、温度  \\
    &              & 1ヶ月 & 付属装置の点検、動作状態、取付状態  \\ 
    &              & 1ヶ月 & その他必要事項  \\ \cline{2-4}
    &計器用変成器    & 1ヶ月 & 外部の損傷、腐食、発錆、変形、汚損、油洩れ、油量、温度、音響、ヒューズの異常  \\
    &              & 1ヶ月 & その他必要事項  \\ \cline{2-4}
    &避雷器         & 1ヶ月 & 外部の損傷、きれつ、ゆるみ、汚損  \\
    &              & 1ヶ月 & その他必要事項  \\ \cline{2-4}
    &配電盤         & 1ヶ月 & 計器の異常、表示札表示灯の異常  \\
    &              & 1ヶ月 & 操作、切換開閉器などの異常  \\
    &              & 1ヶ月 & その他必要事項  \\ \cline{2-4}
    &電力用コンデンサ& 1ヶ月 & 本体外部点検、漏油、汚損、音響、振動  \\
    &              & 1ヶ月 & その他必要事項  \\ \cline{2-4}
    \hline  
配電設備
    &電線及び支持物  & 1ヶ月 & 電線の高さ及び他の工作物樹木との離隔距離  \\
    &              & 1ヶ月 & 標識保護さくの状況  \\ \cline{2-4}
    &ケーブル       & 1ヶ月 & ヘッド、接続箱、分岐箱など接続部の加熱、損傷、腐食及びコンパウンド油漏れ  \\
    &              & 1ヶ月 & 布設部の無断掘削  \\ 
    &              & 1ヶ月 & 標識、他物との離隔距離  \\ \cline{2-4}
    &断路器         & 1ヶ月 & 受けと刃の接触、過熱、変色、ゆるみ  \\
    &              & 1ヶ月 & 汚損、異物付着  \\
    &              & 1ヶ月 & その他必要事項  \\\cline{2-4}
    &遮断器         & 1ヶ月 & 外観点検、汚損、油洩れ、きれつ、過熱、発錆、損傷  \\
    &開閉器類       & 1ヶ月 & 指示、点灯  \\ 
    &              & 1ヶ月 & その他必要事項  \\ \cline{2-4}
    &配電用変圧器   & 1ヶ月 & 本体の外部点検、漏油、損傷、汚損、変形、ゆるみ、発錆、腐食、振動、音響、油量、温度  \\
    &              & 1ヶ月 & 付属装置の点検、動作状態、取付状態  \\ 
    &              & 1ヶ月 & その他必要事項  \\ \cline{2-4}
    &その他付属設備 & 1ヶ月 & 必要により特定範囲のものについて行う  \\ \cline{2-4}
    \hline 
負荷設備
    &電動機その他回転機& 常時 & 運転者が音響、回転、過熱、異臭、給油状況などについて注意する  \\
    &              & 1ヶ月 & 必要により特定範囲のものについて電気担当者が行う  \\ \cline{2-4}
    &電熱乾燥装置     & 常時 & 運転者が温度、変形、損傷などについて注意する  \\
    &              & 1ヶ月 & 接続部変色、過熱、熱線の腐食、取付点検  \\ 
    &              & 1ヶ月 & 必要により特定範囲のものについて電気担当者が行う  \\ \cline{2-4}
    &照明設備        & 常時 & 使用者が異音、汚損、不点、温度、臭気過熱などに注意する  \\\cline{2-4}
    &配線及び配線器具   & 1ヶ月 & 開閉器の点検、湿気、じんあい等に注意、器具の損傷、腐食、分電盤スイッチ、ヒューズの適正及びゆるみ、過熱\\ \cline{2-4}
    \hline  
発電設備
    &原動機関係     & 1ヶ月 & 燃料系統からの油漏及び貯油  \\
    &              & 1ヶ月 & 機関の始動、停止  \\ 
    &              & 1ヶ月 & 始動用空気タンクの圧力その他必要事項  \\ \cline{2-4}
    &発電機関係      & 1ヶ月 &必要により特定範囲のものについて電気担当者が行う  \\ \cline{2-4}
    &蓄電池         & 1ヶ月 & 液面、沈殿物、色相、極板彎曲、隔離板、端子のゆるみ、損傷  \\
    &              & 1ヶ月 & 充電装置の動作状態  \\
    &              & 1ヶ月 & 電池の電圧\\\cline{2-4}\hline  
  \end{tabular}
\end{table}

\begin{table}[b]
\scriptsize
  \caption{定期巡視点検手入項目}
  \label{定期巡視点検手入項目}
  \begin{tabular}{|l|c|p{1cm}|p{10cm}|} \hline
    対象         & & 周期 & 点検箇所と目的  \\ \hline
受変電設備
    &電線及び支持物  & 1年 & 電柱、腕木、がいし、支線、支柱、保護網などの損傷、腐食  \\
    &              & 1年 & 電線取付状態、弛度  \\
    &              & 1年 & その他必要事項  \\ \cline{2-4}
    &ケーブル       & 1年 & ケーブル腐食、きれつ、損傷  \\
    &              & 1年 & その他必要事項  \\ \cline{2-4}
    &断路器         & 1年 & 停止して受けと刃の接触、過熱、ゆるみ、荒れ具合  \\
    &              & 1年 & 汚損、きれつ  \\
    &              & 1年 & フレ止め装置の機能  \\
    &              & 1年 & その他必要事項  \\ \cline{2-4}
    &遮断器         & 1年 & 停止して外部の損傷、腐食、過熱、油量、発錆、変形、ゆるみ  \\ 
    &開閉器類       & 1年 & 操作具合、機構  \\
    &              & 1年 & 付属装置の状態  \\
    &              & 1年 & 油の汚れ、必要によりその特性調査  \\
    &              & 1年 & 接地線接続部  \\
    &              & 1年 & その他必要事項  \\ \cline{2-4}
    &母線          & 1年 & 母線の高さ、たるみ、他物との離隔距離、腐食、損傷過熱  \\ 
    &              & 1年 & 接続部分、クランプ類の腐食、損傷過熱、ゆるみ  \\
    &              & 1年 & がいし類、支持物の腐食、損傷、変形、ゆるみ  \\
    &              & 1年 & その他必要事項  \\\cline{2-4}
    &受電用変圧器   & 1年 & 停止して各部の損傷、腐食、発錆、ゆるみ、変形、きれつ、汚損、油量  \\
    &              & 1年 & 付属装置各部の点検(機能及び状態)  \\ 
    &              & 1年 & 油の汚れ、必要により特性調査  \\ 
    &              & 1年 & 接地線接続部  \\ 
    &              & 1年 & その他必要事項  \\ \cline{2-4}
    &計器用変成器    & 1年 & 停止して各部の損傷、腐食、接触、発錆、ゆるみ、変形、きれつ、汚損、油洩れ、ヒューズの異常  \\
    &              & 1年 & 接地線接続部  \\
    &              & 1年 & その他必要事項  \\ \cline{2-4}
    &避雷器         & 1年 & 外部の損傷、きれつ、ゆるみ、汚損、コンパウンドの異常  \\
    &              & 1年 & 接地線接続部  \\ 
    &              & 1年 & その他必要事項  \\ \cline{2-4}
    &配電盤         & 1年 & 裏面配線の塵埃汚損、損傷、過熱ゆるみ、断線  \\
    &              & 1年 & 接地線接続部  \\\cline{2-4}
    &電力用コンデンサ& 1年 & 外部の損傷、腐食  \\
    &              & 1年 & 接地線接続部  \\ \cline{2-4}
    \hline  
配電設備
    &電線及び支持物  & 1年 & 電柱、腕木、がいし、支線、支柱、保護網などの損傷、腐食  \\
    &              & 1年 & 電線取付状態、弛度  \\
    &              & 1年 & その他必要事項  \\ \cline{2-4}
    &ケーブル       & 1年 & ケーブル腐食、きれつ、損傷  \\
    &              & 1年 & その他必要事項  \\ \cline{2-4}
    &断路器         & 1年 & 停止して受けと刃の接触、過熱、ゆるみ、荒れ具合  \\
    &              & 1年 & 汚損、きれつ  \\
    &              & 1年 & フレ止め装置の機能  \\
    &              & 1年 & その他必要事項  \\ \cline{2-4}
    &遮断器         & 1年 & 停止して外部の損傷、腐食、過熱、油量、発錆、変形、ゆるみ  \\ 
    &開閉器類       & 1年 & 操作具合、機構  \\
    &              & 1年 & 付属装置の状態  \\
    &              & 1年 & 油の汚れ、必要によりその特性調査  \\
    &              & 1年 & 接地線接続部  \\
    &              & 1年 & その他必要事項  \\ \cline{2-4}
    &配電用変圧器   & 1年 & 停止して各部の損傷、腐食、発錆、ゆるみ、変形、きれつ、汚損、油量  \\
    &              & 1年 & 付属装置各部の点検(機能及び状態)  \\ 
    &              & 1年 & 油の汚れ、必要により特性調査  \\ 
    &              & 1年 & 接地線接続部  \\ 
    &              & 1年 & その他必要事項  \\ \cline{2-4}
    &その他付属設備 & 1年 & 母線、がいし、クランプ、支持物などは受変電設備用に準じて行う(停止せず)  \\ \cline{2-4}
    \hline 
負荷設備
    &電動機その他回転機& 3ヶ月 & 音響、振動、温度  \\
    &              & 1年 & 停止して各部の汚損、ゆるみ、損傷伝達装置の異常など外部点検を行う  \\
    &              & 1年 & 制御装置点検  \\
    &              & 1年 & 接地線接続部  \\
    &              & 1年 & その他必要事項  \\ \cline{2-4}
    &電熱乾燥装置     & 1年 & 停止して各部の変形、損傷、ゆるみ可燃物との離隔状況  \\
    &              & 1年 & その他必要事項  \\ \cline{2-4}
    &照明設備        & 1年 & 照明効果、汚損、音響、温度、コンパウンド洩れ  \\
    &              & 1年 & その他必要事項  \\ \cline{2-4}
    &配線及び配線器具   & 1年 & 開閉器、器具との接続、器具の損傷、腐食、分電盤スイッチ、ヒューズの適正及びゆるみ、過熱\\ \cline{2-4}
    \hline  
発電設備
    &原動機関係     & 1年 & 機関主要部分の分解、点検  \\\cline{2-4}
    &発電機関係     & 3ヶ月 & 音響、振動、温度  \\
    &              & 1年 & 停止して各部の汚損、ゆるみ、損傷伝達装置の異常など外部点検を行う  \\
    &              & 1年 & 制御装置点検  \\
    &              & 1年 & 接地線接続部  \\
    &              & 1年 & その他必要事項  \\ \cline{2-4}

    &蓄電池         & 1年 & 木台、がいしの腐食、損傷、耐酸塗料のはくり  \\
    &              & 1年 & 床面の腐食、損傷  \\
    &              & 1年 & その他必要事項\\\cline{2-4}\hline  
  \end{tabular}
\end{table}

\begin{table}[b]
\scriptsize
  \caption{精密点検手入項目}
  \label{精密点検手入項目}
  \begin{tabular}{|l|c|p{1cm}|p{10cm}|} \hline
    対象         & & 周期 & 点検箇所と目的  \\ \hline
受変電設備
    &電線及び支持物  & 3-5年 & 必要により特定対象を定めて行う(点検箇所、部位は定期巡視点検より抜粋)  \\\cline{2-4}
    &ケーブル       & 5年 & 必要により特定対象を定めて行う(点検箇所、部位は定期巡視点検より抜粋)   \\
    &              & 3-5年 & 地盤沈下の影響  \\ \cline{2-4}
    &遮断器         & 2年又は一定の遮断回数による & 停止して内部について接触子の荒れ具合、ゆるみ、変形、焼損、損傷  \\ 
    &開閉器類       & 〃    & 操作機構及び付属装置の各部点検  \\
    &              & 〃    & 遮断速度測定(開極投入時間最小動作電圧及び電流の測定を含む)  \\
    &              & 〃    & その他必要事項  \\ \cline{2-4}
    &母線          & 3年 & 必要により特定対象を定めて行う(点検箇所、ねらいは定期巡視点検より抜粋)  \\\cline{2-4}
    &受電用変圧器   & 5-10年 & 停止して内部について点検(コイル接続部、リード線、鉄心、その他各部)  \\
    &              & 5年 & 付属装置及び機器の内部点検  \\ 
    &              & 5年 & その他必要事項  \\ \cline{2-4}
    &計器用変成器    & 3年 & 油入式について、停止して内部の点検  \\
    &              & 2年 & 必要により油の汚れ及び特性調査  \\
    &              & 3年 & その他必要事項  \\ \cline{2-4}
    &配電盤         & 2年 & 停止して各部の損傷、過熱、ゆるみ断線、接触、脱落  \\
    &              & 2年 & 端子、配線符号  \\
    &              & 2年 & 接地線接続部  \\\cline{2-4}
    \hline  
配電設備
    &電線及び支持物  & 3-5年 & 必要により特定対象を定めて行う(点検箇所、部位は定期巡視点検より抜粋)  \\\cline{2-4}
    &ケーブル       & 5年 & 必要により特定対象を定めて行う(点検箇所、部位は定期巡視点検より抜粋)   \\
    &              & 3-5年 & 地盤沈下の影響  \\ \cline{2-4}
    &遮断器         & 2年又は一定の遮断回数による & 停止して内部について接触子の荒れ具合、ゆるみ、変形、焼損、損傷  \\ 
    &開閉器類       & 〃    & 操作機構及び付属装置の各部点検  \\
    &              & 〃    & 遮断速度測定(開極投入時間最小動作電圧及び電流の測定を含む)  \\
    &              & 〃    & その他必要事項  \\ \cline{2-4}
    &配電用変圧器   & 5-10年 & 停止して内部について点検(コイル接続部、リード線、鉄心、その他各部)  \\
    &              & 5年 & 付属装置及び機器の内部点検  \\ 
    &              & 5年 & その他必要事項  \\ \cline{2-4}
    &その他付属設備 & 3年 & 必要により特定対象を定めて行う(この場合停止して点検する)  \\
    &              & 3年 & その他必要事項  \\ \cline{2-4}
    \hline 
負荷設備
    &電動機その他回転機& 3年 & 必要により特定対象を定めて行う温度上昇等を考慮し内部分解点検、コイル、軸受、通風、付属装置などの手入  \\
    &              & 3年 & 温度上昇等を考慮し、回転子引出掃除  \\
    &              & 3年 & その他必要事項  \\ \cline{2-4}
    &電熱乾燥装置     & 3年 & 必要により特定対象を定めて行う(点検箇所、部位は定期に準じて内部点検を行う)  \\\cline{2-4}
    &配線及び配線器具   & 2年 & 許容電流と負荷電流との確認\\ \cline{2-4}
    \hline  
発電設備
    &原動機関係     & 3年又は一定の運転時間による & 内燃機関の分解、点検、測定  \\\cline{2-4}
    &発電機関係     & 3年 & 必要により特定対象を定めて行う温度上昇等を考慮し内部分解点検、コイル、軸受、通風、付属装置などの手入  \\
    &              & 3年 & 温度上昇等を考慮し、回転子引出掃除  \\
    &              & 3年 & その他必要事項  \\ \cline{2-4}
    &蓄電池         & 3年 & 充電装置の内部点検  \\
    &              & 3年 & 必要により対象を定めて行う\\\cline{2-4}\hline  
  \end{tabular}
\end{table}

\begin{table}[b]
\scriptsize
  \caption{測定項目}
  \label{測定項目}
  \begin{tabular}{|l|c|p{1cm}|p{10cm}|} \hline
    対象         & & 周期 & 点検箇所と目的  \\ \hline
受変電設備
    &電線及び支持物  & 1年 & 絶縁抵抗測定  \\\cline{2-4}
    &ケーブル       & 1年 & 絶縁抵抗測定   \\
    &              & 1年 & 接地抵抗測定  \\ \cline{2-4}
    &断路器         & 1年 & 接地抵抗測定  \\ \cline{2-4}
    &遮断器         & 1年 & 絶縁抵抗測定  \\ 
    &開閉器類       & 1年 & 接地抵抗測定  \\
    &              & 3年 & 絶縁油試験  \\
    &              & 不定期 & 必要により動作特性  \\ \cline{2-4}
    &母線          & 1年 & 絶縁抵抗測定  \\\cline{2-4}
    &受電用変圧器   & 1年 & 絶縁抵抗測定  \\
    &              & 1年 & 接地抵抗測定  \\ 
    &              & 3年 & 必要により絶縁油試験 \\\cline{2-4}
    &計器用変成器   & 1年 & 絶縁抵抗測定  \\
    &              & 1年 & 接地抵抗測定  \\\cline{2-4}
    &避雷器         & 1年 & 絶縁抵抗測定  \\
    &              & 1年 & 接地抵抗測定  \\\cline{2-4}
    &配電盤         & 1年 & 絶縁抵抗測定  \\
    &              & 1年 & 接地抵抗測定  \\
    &              & 2年 & 保護継電器の動作特性  \\
    &              & 2年 & 必要により計器校正、シーケンス試験  \\\cline{2-4}
    &電力用コンデンサ& 1年 & 絶縁抵抗測定  \\
    &              & 1年 & 接地抵抗測定  \\\cline{2-4}
    \hline  
配電設備
    &電線及び支持物  & 1年 & 絶縁抵抗測定  \\\cline{2-4}
    &ケーブル       & 1年 & 絶縁抵抗測定   \\
    &              & 1年 & 接地抵抗測定  \\ \cline{2-4}
    &断路器         & 1年 & 接地抵抗測定  \\ 
    &遮断器         & 1年 & 絶縁抵抗測定  \\ 
    &開閉器類       & 3年 & 絶縁油試験  \\
    &              & 不定期 & 必要により動作特性  \\ \cline{2-4}
    &配電用変圧器   & 1年 & 絶縁抵抗測定  \\
    &              & 1年 & 接地抵抗測定  \\ 
    &              & 3年 & 必要により絶縁油試験 \\\cline{2-4}
    &その他付属設備   & 1年 & 絶縁抵抗測定  \\
    &              & 1年 & 接地抵抗測定  \\\cline{2-4}
    \hline 
負荷設備
    &電動機その他回転機& 1年 & 絶縁抵抗測定  \\
    &               & 1年 & 接地抵抗測定  \\
    &               & 1年 & 必要により特性試験  \\ \cline{2-4}
    &電熱乾燥装置     & 1年 & 絶縁抵抗測定  \\
    &               & 1年 & 接地抵抗測定  \\ \cline{2-4}
    &照明設備        & 1年 & 絶縁抵抗測定  \\
    &               & 1年 & 接地抵抗測定  \\
    &               & 3年 & 必要により照明測定  \\ \cline{2-4}
    &配線及び配線器具 & 1年 & 絶縁抵抗測定  \\
    &               & 1年 & 接地抵抗測定  \\
    &               & 1年 & 必要により配線用遮断器及び漏電遮断器の特性試験  \\ \cline{2-4}
    \hline  
発電設備
    &発電機関係     & 1年 & 絶縁抵抗測定  \\
    &               & 1年 & 接地抵抗測定  \\
    &              & 3年 & 継電器試験  \\ \cline{2-4}
    &蓄電池         & 1ヶ月 & 比重測定  \\
    &              & 1ヶ月 & 液温測定  \\
    &              & 1ヶ月 & 電圧測定  \\
    &              & 1年 & 絶縁抵抗測定(充電装置)\\\cline{2-4}\hline  
  \end{tabular}
\end{table}

\article{法定事業者検査の体制}
\term 法定事業者検査は、電気主任技術者の監督の下、別途定める必要な事項をあらかじめ決定した上で行うものとする。
\article{事故の再発防止}
\term 事故その他異常が発生した場合には、必要に応じ臨時に精密検査を行い、その原因を究明し、再発防止に遺憾のないよう措置するものとする。

\subsection{運転又は操作}
\article{運転又は操作等}
電気工作物の運転または操作の基準は、別に定める細則によるものとする。
\term 前項の操作の順序及び方法については、受電室その他見やすい場所に掲示しておくものとする。
\term 受電用遮断器の操作に当たっては、必要に応じて電気事業者との連絡を密にし、電気工作物の運用に支障が生じないように努めるものとする。

\subsection{災害対策}
\article{防災体制}
台風、洪水、地震、火災、その他の非常災害に備えて、電気工作物に関する保安を確保するために、防災思想を従業者に徹底し、応急資材を備蓄するとともに、災害発生時の措置に関する体制をあらかじめ整備し、並びに当事業場外関係機関との協力体制及び連携体制を整備しておくものとする。
\term 電気主任技術者は、非常災害発生時において、電気工作物に関する保安を確保するための指揮監督を行う。
\term 電気主任技術者は、災害等の発生に伴い危険と認められるときは、直ちに当該範囲の送電を停止することができるものとする。

\subsection{記録}

\article{記録等}
電気工作物の工事等に関する記録は次の各号について記録し、これを保存するものとする。また、記録内容については、別に細則で定めるものとする。
\begin{enumerate}  
\itm 巡視、点検及び測定記録
\itm 電気事故の記録
\itm 補修工事記録
\end{enumerate}
\term 主要電気機器の補修記録は別表第3に定める設備台帳により記録し、必要な期間保存するものとする。
\term 法定事業者検査の記録は、別表第4に定めるところにより記録し、必要な期間保存するものとする。

\subsection{責任の分界}

\article{責任の分界点}
電気事業者との保安上の責任及び財産分界点は、需給契約書に基づく責任分界点  とする。
\article{需要設備の構内}
当事業場の需要設備の構内は別図(需要設備の構内図)に示すとおりとする。

\subsection{整備その他}

\article{危険の表示}
受電室その他高圧電気工作物が設置されている場所等であって、危険のおそれのあるところには、人の注意を喚起する表示を設けなければならない。

\article{測定器具類の整備}
電気工作物の保安上必要とする測定器具類は常に整備し、これを適正に保管しなければならない。

\article{図面、書類の整備}
電気工作物に関する結線図、系統図、配線図、主要機器関係図、設計図、仕様書、取扱い説明書等については整備し、必要な期間保存しなければならない。

\article{手続き書類等の整備}
関係官庁、電気事業者等に提出した書類及び図面その他主要な文書については、その写しを必要な期間保存しなければならない。


\begin{figure}[h]
\centerline{%
    \ROOT{総括管理者}{%
    \BRANCH{電気主任技術者}{%
        \LEAF{連絡責任者}%
      }%
    }%
}
\caption{組織図}
\label{fig:soshiki}
\end{figure}

\vspace{1cm}
\subparagraph{附則}
この規程は、令和元年12月25日から施行する。

%\begin{tabular}
%\end{tabular}

\end{document}
 
%


